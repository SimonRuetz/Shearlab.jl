%%% Mathematische Makros

%%% Umgebungen

%%% Allgemeines
% flexibler Klammerbefehl, #1=links, #2=rechts, #3=Größe (normal, auto, opauto, \[sizecmd]), #4=Inhalt
\newcommandx{\braces}[4]{%
\ifstrequal{#3}{normal}{#1#4#2}{%
\ifstrequal{#3}{auto}{\left#1#4\right#2}{%
\ifstrequal{#3}{opauto}{\opleft#1#4\opright#2}{%
#3#1#4#3#2}}}%
}
% Klammern für Operatoren (entfernt das Spacing vor und nach der Klammer)
\newcommand{\opleft}[1]{\mathopen{}\left#1}
\newcommand{\opright}[1]{\right#1\mathclose{}}
% Anmerkung über Operatoren
\newcommandx{\opannot}[3][3=\downarrow]{\stackrel{\mathclap{\substack{#1 \\ #3 \vspace{2pt}}}}{#2}}
% Anmerkung vor einer Zeile
\newcommandx{\lineannot}[3][3=\rightarrow]{\mathllap{\boxed{\text{\textsmaller{#1}}} #3} #2}
% Anmerkung vor einer Zeile (mehrzeilig)
\newcommandx{\multilineannot}[4][4=\rightarrow]{\mathllap{\boxed{\parbox{#1}{\RaggedRight\textsmaller{#2}}} #4} #3}
% Anmerkung zwischen zwei Zeilen
\newcommand{\interannot}[1]{\boxed{\text{\textsmaller{#1}}}} 
% Anmerkung zwischen zwei Zeilen (mehrzeilig)
\newcommand{\multiinterannot}[2][.5\textwidth]{\boxed{\parbox{#1}{\RaggedRight\textsmaller{#2}}}} 

%%% Zahlenmengen
\newcommand{\N}{\mathbb{N}} % natürliche Zahlen
\newcommand{\Nzero}{\mathbb{N}_0} % natürliche Zahlen mit Null
\newcommand{\Z}{\mathbb{Z}} % ganze Zahlen
\newcommand{\Q}{\mathbb{Q}} % rationale Zahlen
\newcommand{\R}{\mathbb{R}} % reelle Zahlen
\newcommand{\Rpos}{\mathbb{R}_{>0}} % positive reelle Zahlen
\newcommand{\C}{\mathbb{C}} % komplexe Zahlen
\newcommand{\K}{\mathbb{K}} % reelle oder komplexe Zahlen
%\newcommand{\H}{{H}} % reelle oder komplexe Zahlen

%%% Aussagenlogik
% \renewcommand{\mid}{:} % Trennung der Bedingung bei der Angabe von Mengen
\renewcommand{\iff}{\Leftrightarrow} % genau dann wenn
\renewcommand{\implies}{\Rightarrow} % impliziert
\newcommand{\suchthat}[1][normal]{\ifstrequal{#1}{normal}{\mid}{#1|}}

%%% Mengen und Topologie
\newcommand{\setcompl}[1]{#1^c} % Mengenkomplement
\newcommand{\cardinality}{\#}
\newcommand{\union}{\cup} % Vereinigung
\newcommand{\bigunion}{\bigcup} % Vereinigung
\newcommand{\intersec}{\cap} % Schnittmenge
\newcommand{\bigintersec}{\bigcap} % Schnittmenge
\newcommand{\boundary}[1]{\partial#1} % Rand einer Menge
\newcommand{\clos}[1]{\overline{#1}} % topologischer Abschluss
\newcommand{\interior}[1]{\mathring{#1}} % das innere eine Menge
\newcommand{\dist}[2]{\operatorname{dist}(#1, #2)} % Distanz zwischen zwei Mengen
\newcommandx{\intvcl}[3][1=normal]{\braces{[}{]}{#1}{#2, #3}} % abgeschlossenes Intervall
\newcommandx{\intvop}[3][1=normal]{\braces{(}{)}{#1}{#2, #3}} % offenes Intervall
\newcommandx{\intvclop}[3][1=normal]{\braces{[}{)}{#1}{#2, #3}} % halboffenes Intervall (rechts)
\newcommandx{\intvopcl}[3][1=normal]{\braces{(}{]}{#1}{#2, #3}} % halboffenes Intervall (links)
\newcommand{\indcoeff}[1]{\mathds{1}_{#1}} % Indikatorfunktion für Indizes von Koeffizienten

%%% Analysis
\newcommand{\dotarg}{\ensuremath{\raisebox{.15ex}{$\scriptstyle [\cdot]$}}}
\DeclareMathOperator*{\argmin}{argmin} % argmin
\DeclareMathOperator*{\argmax}{argmax} % argmax
\newcommandx{\abs}[2][1=normal]{\braces{\lvert}{\rvert}{#1}{#2}} % Absolutbetrag
\newcommand{\conj}[1]{\overline{#1}} % komplexe Konjugation
\newcommandx{\ceil}[2][1=normal]{\braces{\lceil}{\rceil}{#1}{#2}} % Ceil
\newcommandx{\floor}[2][1=normal]{\braces{\lfloor}{\rfloor}{#1}{#2}} % Floor
\newcommandx{\round}[2][1=normal]{\braces{[}{]}{#1}{#2}} % Runden
\newcommandx{\der}[1]{D^{#1}} % Differentialoperator (#1 = Multiindex)
\newcommandx{\partder}[4][1={},4={}]{\frac{\partial^{#4} #2}{\partial #3^{#4}}\ifargdef{#1}{\Big|_{#1}}} % partielle Abbleitung
\newcommandx{\integ}[4][1={},2={}]{\int_{#1}^{#2} #3 \, #4} % Integral (#1op = untere Grenze, #2op = obere Grenze, #3 = Integrant, #4 = Differentialform)
\newcommandx{\asympffaster}[2][1=normal]{o\braces{(}{)}{#1}{#2}} % Asymptotisch echt schneller
\newcommandx{\asympfaster}[2][1=normal]{O\braces{(}{)}{#1}{#2}} % Asymptotisch schneller
\newcommandx{\asympeq}[2][1=normal]{\Theta\braces{(}{)}{#1}{#2}} % Asymptotische Gleichheit
\newcommandx{\asympsslower}[2][1=normal]{\omega\braces{(}{)}{#1}{#2}} % Asymptotisch echt langsamer
\newcommandx{\asympslower}[2][1=normal]{\Omega\braces{(}{)}{#1}{#2}} % Asymptotisch langsamer
\newcommandx{\measure}[2][1=normal]{\braces{\lvert}{\rvert}{#1}{#2}} % Maß einer Menge
\DeclareMathOperator{\supp}{supp} % Träger

%%% Lineare Algebra und Funktionalanalysis
\DeclareMathOperator{\Id}{Id} % Identität
\newcommand{\matr}[1]{\begin{bmatrix} #1 \end{bmatrix}} % Matrix
\newcommandx{\norm}[2][1=normal]{\braces{\|}{\|}{#1}{#2}} % Norm
\renewcommandx{\sp}[3][1=normal]{\braces{\langle}{\rangle}{#1}{#2, #3}} % Skalarprodukt
\newcommand{\adj}[1]{{#1}^\ast} % Adjungierter Operator
\newcommandx{\End}[2][2={}]{\mathcal{L}\opleft( #1 \ifargdef{#2}{, #2} \opright)} % Endomorphismen
\newcommand{\orthsum}{\oplus} % orthogonale Summe
\newcommand{\orthcompl}[1]{{#1}^\perp} % orthogonale Summe
\newcommand{\tensprod}{\otimes} % Tensorprodukt
\DeclareMathOperator{\ran}{ran} % Bild
% \DeclareMathOperator{\ker}{ker} % Kern
\DeclareMathOperator{\spann}{\operatorname{span}} % Spann
\newcommand{\T}{\top} % Transpositionssymbol

%%% Funktionenräume
\newcommand{\almostev}{\text{f.ü.}} % fast überall
\newcommand{\indset}[1]{\chi_{#1}} % Indikatorfunktion für Mengen
% Lebesgueräume (1 - Menge. 2 - Exponent)
\newcommandx{\Leb}[3][1={},3=normal]{L^{#2}\ifargdef{#1}{\braces{(}{)}{#3}{#1}}{}}
\newcommandx{\Lebnorm}[4][1=normal,3={2},4={}]{\norm[#1]{#2}_{\Leb[#4]{#3}}}
% Folgenräume (1 - Menge, 2 - Exponent)
\renewcommandx{\l}[3][1={},3=normal]{\ell^{#2}\ifargdef{#1}{\braces{(}{)}{#3}{#1}}}
\newcommandx{\lnorm}[4][1=normal,3={2},4={}]{\norm[#1]{#2}_{#3}}
% Räume differenzierbarer Funktionen (#1op=Menge, #2Differenzierbarkeit, #3op=Subscript)
\newcommandx{\Smooth}[4][1={},3={},4=normal]{C_{#3}^{#2}\ifargdef{#1}{\braces{(}{)}{#4}{#1}}} 
% Raum der Schwartzfunktionen
\newcommandx{\Schwartz}[2][1={},2=normal]{\mathscr{S}\ifargdef{#1}{\braces{(}{)}{#2}{#1}}} 
\newcommandx{\Schwartzpoly}[2][1=normal]{\braces{\langle}{\rangle}{#1}{\abs[#1]{#2}} }
\newcommand{\Schwartznorm}[3]{C_{#1,#2}(#3)}
% temperierte Distributionen
\newcommandx{\Tempdistr}[2][1={},2=normal]{\mathscr{S}'\ifargdef{#1}{\braces{(}{)}{#2}{#1}}}
% Setze die Funktion #2 in die Distribution #1 ein
\newcommandx{\distrinp}[3][1=normal]{\braces{(}{)}{#1}{#2, #3}} 
\newcommand{\Linedistr}[1][]{\mathfrak{L}\ifargdef{#1}{_{#1}}{}} % Liniendistribution
\newcommand{\conv}{\ast} % Faltungsoperator
% Fouriertransformation (Hut-Notation) TODO: 'long' als Wert
\newcommandx{\ft}[3][1=default,2=auto]{
\ifstrequal{#1}{default}{\widehat{#3}}{
\ifstrequal{#1}{long}{{\braces{(}{)}{#2}{#3}}^{\wedge}}{}}}
\newcommand{\ftop}{\mathcal{F}} % FT als Operator
% inv. Fouriertransformation (Hut-Notation) TODO: widecheck
\newcommandx{\ift}[3][1=default,2=auto]{
\ifstrequal{#1}{default}{\check{#3}}{
\ifstrequal{#1}{long}{{\braces{(}{)}{#2}{#3}}^{\vee}}{}}}
\newcommand{\iftop}{\mathcal{F}^{-1}} % inv. FT als Operator

%%% Sonstiges
\newcommand{\sinc}{\operatorname{sinc}} % sinc
\newcommand{\eps}{\varepsilon} % Abkürzung für epsilon


% % Aufzählungen
% % Nummerierung mit kleinen römischen Zahlen
% \newcounter{romancnt}
% \newenvironment{romanlist}
%  {\setcounter{romancnt}{1}
%  \begin{list}{(\roman{romancnt})}{\leftmargin=1.5em \itemindent=0em}
%  \usecounter{romancnt}}
%  {\end{list}}
% 
% % Nummerierung mit kleinen arabischen Ziffern
% \newcounter{numbercnt}
% \newenvironment{numberlist}
%  {\setcounter{numbercnt}{1}
%  \begin{list}{\arabic{numbercnt})}{\leftmargin=1.5em \itemindent=0em}
%  \usecounter{numbercnt}}
%  {\end{list}}
% 
% % Nummerierung mit kleinen Buchstaben
% \newcounter{alphcnt}
% \newenvironment{alphalist}
%  {\setcounter{alphcnt}{1}
%  \begin{list}{\alph{alphcnt})}{\leftmargin=1.5em \itemindent=0em}
%  \usecounter{alphcnt}}
%  {\end{list}}
% 
% % STANDARDMAKROS
% % Zahlenmengen und Symbole
% \newcommand{\N}{\mathbb{N}}
% \newcommand{\Z}{\mathbb{Z}}
% \newcommand{\Q}{\mathbb{Q}}
% \newcommand{\R}{\mathbb{R}}
% \newcommand{\C}{\mathbb{C}}
% \newcommand{\K}{\mathbb{K}}
% \newcommand{\Pot}{\mathcal{P}}
% \newcommand{\eps}{\varepsilon}
% \renewcommand{\phi}{\varphi}
% % Pfeile
% \newcommand{\lra}{\Leftrightarrow}
% \newcommand{\lraq}{\ \Leftrightarrow \ }
% \newcommand{\ra}{\Rightarrow}
% \newcommand{\raq}{\ \Rightarrow \ }
% \newcommand{\la}{\Leftarrow}
% \newcommand{\laq}{\ \Leftarrow \ }
% \newcommand{\lral}{\Longleftrightarrow}
% \newcommand{\lralq}{\ \Longleftrightarrow \ }
% \newcommand{\ral}{\Longrightarrow}
% \newcommand{\ralq}{\ \Longrightarrow \ }
% \newcommand{\lal}{\Longleftarrow}
% \newcommand{\lalq}{\ \Longleftarrow \ }
% % Operatoren und Aehnliches
% \newcommand{\ex}{\exists}
% \newcommand{\fa}{\forall}
% \newcommand{\com}{\complement}
% \renewcommand{\Im}{\operatorname{Im}}
% \renewcommand{\Re}{\operatorname{Re}}
% \newcommand{\op}[1]{\operatorname{#1}}
% \newcommand{\card}{\operatorname{card}}
% \newcommand{\toper}[2][=]{\stackrel{\text{#2}}{#1}}
% \newcommand{\eqoper}[2][=]{\stackrel{\substack{#2}}{#1}}
% \newcommand{\inv}[1]{\frac{1}{#1}}
% \newcommand{\argmin}[1]{\underset{#1}{\operatorname{argmin}}}
% \newcommand{\argmax}[1]{\underset{#1}{\operatorname{argmax}}}
% \renewcommand{\sp}[2]{\left\langle #1 , #2 \right\rangle}
% % Matrizen
% \newcommand{\matr}[1]{\begin{bmatrix} #1 \end{bmatrix}}
% % Sonstiges
% \newcommand{\relphantom}[1]{\mathrel{\phantom{#1}}} 
% \newcommand{\flr}[1]{\begin{flushright}{#1}\end{flushright}}
% \newenvironment{indented}[1]%
% {\begin{list}{}%
%          {\setlength{\leftmargin}{#1}}%
%          \item[]%
% }%
% {\end{list}}
% \newcommand{\PBox}{\hfill $_\blacksquare$}
% 
% \newcommand{\amatrix}[2]{\left[ \begin{array}{#1} #2 \end{array} \right]}
% 
% \newcommand{\norm}[1]{\left\|#1\right\|}
% \newcommand{\abs}[1]{\left\lvert #1 \right\rvert}
% \newcommand{\clos}[1]{\overline{#1}}
% % \renewcommand{\span}{\operatorname{span}}
% \newcommand{\conj}[1]{\overline{#1}}
% \newcommand{\ran}{\op{ran}}
% 
% \renewcommand{\H}{\mathcal{H}}
% \newcommand{\weak}{\rightharpoonup}
% \renewcommand{\l}{\ell}
% \renewcommand{\L}{\mathcal{L}}
% \newcommand{\spec}{\sigma}
% \newcommand{\adj}[1]{{#1}^\ast}
% \newcommand{\dom}{\op{dom}}
% \newcommand{\AC}{\op{AC}}
% \newcommand{\FT}{\mathcal{F}}
% \newcommand{\ftfact}{\frac{1}{(2\pi)^{n/2}}}
% \newcommand{\schw}{\mathcal{S}}
