% ------------- Packages

%--------------------------------------------------------------------------------
%---- Julia highlighting settings
%--------------------------------------------------------------------------------
\usepackage{beramono}
\usepackage{listings}

\lstdefinelanguage{julia}{morekeywords={abstract,break,case,catch,const,continue,do,else,elseif,%
      end,export,false,for,function,immutable,import,importall,if,in,%
      macro,module,otherwise,quote,return,switch,true,try,type,typealias,%
      using,while},%
   sensitive=true,%
   morecomment=[l]\#,%
   morecomment=[n]{\#=}{=\#},%
   morestring=[s]{"}{"},%
   morestring=[m]{'}{'},%
}[keywords,comments,strings]%


% \usepackage{default}

%%% Doc: http://mirror.informatik.uni-mannheim.de/pub/mirrors/tex-archive/macros/latex/contrib/xargs/xargs.pdf
% More than one optional argument
\usepackage{xargs}

%%%
% IfThenElse
\usepackage{ifthen}

%%% Doc: ftp://ftp.mpi-sb.mpg.de/pub/tex/mirror/ftp.dante.de/pub/tex/macros/latex/contrib/etoolbox/etoolbox.pdf
% Programming in LaTeX
\usepackage{etoolbox}

%%% Doc: ftp://tug.ctan.org/pub/tex-archive/macros/latex/required/babel/babel.pdf
% Language setting
\usepackage[
%	german,
% 	ngerman,
	english,
%	french,
]{babel}

%%% Doc: ftp://tug.ctan.org/pub/tex-archive/macros/latex/required/graphics/grfguide.pdf
% Graphics
\usepackage[%
	%final,
	%draft % do not include images (faster)
]{graphicx}

\usepackage{array}
\usepackage{multicol}
\usepackage{wrapfig}

% \usepackage{memh­fixc}
% \usepackage{memoir}

\usepackage{relsize}

\usepackage{mdframed}

% \usepackage{enumitem}

%%%
% This package is used to create two-sided presentations with notes
\usepackage{pgfpages}

\usepackage{tikz}
\usetikzlibrary{arrows,shapes,calc}
\newcommand{\tikzmark}[2]{\tikz[overlay,remember picture] \node[minimum width=1.5em] (#1) {#2};}
\newcommand{\tikzcoord}[1]{\tikz[overlay,remember picture] \coordinate (#1);}
\makeatletter
\protected\def\tikz@nonactivecolon{\ifmmode\mathrel{\mathop\ordinarycolon}\else:\fi} 
\makeatother

\newcommand*\circled[1]{\tikz[baseline=(char.base)]{
            \node[shape=circle,draw,inner sep=2pt] (char) {#1};}}


% Put text somewhere on the slide
% \def\Put(#1,#2)#3{\leavevmode\makebox(0,0){\put(#1,#2){#3}}}
\usepackage[absolute,overlay]{textpos}

%%% Doc: ftp://tug.ctan.org/pub/tex-archive/macros/latex/contrib/mh/doc/mathtools.pdf
% Erweitert amsmath und behebt einige Bugs
\usepackage[fixamsmath,disallowspaces]{mathtools}
% Formelnummern nur anzeigen, wenn auch eine Referenz existiert
\mathtoolsset{showonlyrefs}
\mathtoolsset{centercolon=true}
% \usepackage{autonum}

\usepackage{mathrsfs}

\usepackage{bm}

\usepackage{dsfont}
\usepackage{animate}
\usepackage{xcolor}
\usepackage{mdframed}

% ------------- Definitions

%%% Programming
% Falls #2 definiert/nichtleer ist, so schreibe #3, sonst #1
\newcommand{\ifargdef}[3][{}]{\ifthenelse{\equal{#2}{}}{#1}{#3}}

%%% Colors
\definecolor{red}{rgb}{1,0,0}
\definecolor{blue}{rgb}{0,0,1}
\definecolor{green}{rgb}{0.2,0.6,0.15}
\definecolor{darkgreen}{rgb}{0,0.5,0}
\definecolor{lightblue}{rgb}{0,0.5,1}
\definecolor{white}{rgb}{1,1,1}
\definecolor{bluegreen}{rgb}{0,0.5,0.5}
\definecolor{violet}{rgb}{0.5,0,0.5}
\definecolor{ZurichBlue}{rgb}{.35,.35,.73}
\definecolor{TUred}{rgb}{0.7,0,0}
% \definecolor{red}{rgb}{1,0,0}
% \definecolor{blue}{rgb}{0,0,1}
% \definecolor{green}{rgb}{0.2,0.6,0.15}
% \definecolor{darkgreen}{rgb}{0,0.5,0}
% \definecolor{lightblue}{rgb}{0,0.5,1}
% %\definecolor{white}{rgb}{0,0,0}
% \definecolor{bluegreen}{rgb}{0,0.5,0.5}
% \definecolor{violet}{rgb}{0.5,0,0.5}
% \definecolor{ZurichBlue}{rgb}{.35,.35,.73}

\setbeamercolor{highlightbox}{fg=black,bg=ZurichBlue}

% no frame counting in appendix, use together with 'noframenumbering' as frame option
\makeatletter
\preto{\appendix}{%
  \patchcmd{\beamer@continueautobreak}{\refstepcounter{framenumber}}{}{}{}}
\makeatother

% Keep the frame title, if there is a framebreak
\setbeamertemplate{frametitle continuation}{}
% Break frame within a theorem
\newcommand*{\theorembreak}{\usebeamertemplate{theorem end}\framebreak\usebeamertemplate{theorem begin}}
% Definitions highlighting
\newcommand<>{\define}[1]{{\color#2{green} #1}}

\newcommand{\citeref}[1]{{\tiny{\color{gray}#1}}}

\newtheorem{algorithm}{Algorithm}
\newtheorem{proposition}{Proposition}


% ------------- Settings

\mode<presentation>{
	\usetheme{Boadilla}
% 	\useoutertheme{infolines}
% 	\useinnertheme{rounded}
	\usecolortheme{beaver}
	\setbeamercovered{transparent}
	
% Make the default 'red' of beaver darker
	\definecolor{darkred}{rgb}{.65,0,0}
	\setbeamercolor{structure}{fg=ZurichBlue}
	
	\setbeamercolor{block title}{use=structure,fg=white,bg=darkred!80!white}
	\setbeamercolor{block body}{use=structure,fg=black,bg=gray!20!white}
	
	
	\setbeamercolor{frametitle}{bg=gray!20!white}
	
	% Frame title with logo in the right corner
	\setbeamertemplate{frametitle}
	{
		\nointerlineskip
		\begin{beamercolorbox}[sep=0.3cm,wd=\paperwidth]{frametitle}
			\vbox{}\vskip-1ex%
			\strut\insertframetitle\strut \hfill 
			\vskip-1.5ex%
		\end{beamercolorbox}
	}

% 	\setbeamertemplate{footline}[default]
% 	\setbeameroption{show notes on second screen}
% 	\setbeameroption{show notes}
	\setbeamercovered{invisible}
}

\beamertemplatenavigationsymbolsempty

\makeatletter
% add a macro that saves its argument
\newcommand{\footlineextra}[1]{\gdef\insertfootlineextra{#1}}
\newbox\footlineextrabox

% add a beamer template that sets the saved argument in a box.
% The * means that the beamer font and color "footline extra" are automatically added. 
\defbeamertemplate*{footline extra}{default}{
    \begin{beamercolorbox}[ht=2.25ex,dp=1ex,leftskip=\Gm@lmargin]{footline extra}
    \insertfootlineextra
    %\par\vspace{2.5pt}
    \end{beamercolorbox}
}

\addtobeamertemplate{footline}{%
    % set the box with the extra footline material but make it add no vertical space
    \setbox\footlineextrabox=\vbox{\usebeamertemplate*{footline extra}}
    \vskip -\ht\footlineextrabox
    \vskip -\dp\footlineextrabox
    \box\footlineextrabox%
}
{}

% patch \begin{frame} to reset the footline extra material
\let\beamer@original@frame=\frame
\def\frame{\gdef\insertfootlineextra{}\beamer@original@frame}
\footlineextra{}
\makeatother

\newcommand<>{\uncoverubrace}[2]{%
  \onslide#3 \underbrace{ \onslide<1->%
  #1%
  \onslide#3 }_{#2} \onslide<1->%
}

\setbeamertemplate{itemize items}[triangle]
